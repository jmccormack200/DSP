%%%%%%%%%%%%%%%%%%%%%%%%%%%%%%%%%%%%%%%%%
% Short Sectioned Assignment
% LaTeX Template
% Version 1.0 (5/5/12)
%
% This template has been downloaded from:
% http://www.LaTeXTemplates.com
%
% Original author:
% Frits Wenneker (http://www.howtotex.com)
%
% License:
% CC BY-NC-SA 3.0 (http://creativecommons.org/licenses/by-nc-sa/3.0/)
%
%%%%%%%%%%%%%%%%%%%%%%%%%%%%%%%%%%%%%%%%%

%----------------------------------------------------------------------------------------
%	PACKAGES AND OTHER DOCUMENT CONFIGURATIONS
%----------------------------------------------------------------------------------------

\documentclass[paper=letterpaper, fontsize=11pt]{scrartcl} % A4 paper and 11pt font size

\usepackage[T1]{fontenc} % Use 8-bit encoding that has 256 glyphs
%\usepackage{fourier} % Use the Adobe Utopia font for the document - comment this line to return to the LaTeX default
\usepackage[english]{babel} % English language/hyphenation
\usepackage{amsmath,amsfonts,amsthm} % Math packages

\usepackage{lipsum} % Used for inserting dummy 'Lorem ipsum' text into the template

\usepackage{sectsty} % Allows customizing section commands
\allsectionsfont{\centering \normalfont\scshape} % Make all sections centered, the default font and small caps

\usepackage{fancyhdr} % Custom headers and footers
\pagestyle{fancyplain} % Makes all pages in the document conform to the custom headers and footers
\fancyhead{} % No page header - if you want one, create it in the same way as the footers below
\fancyfoot[L]{} % Empty left footer
\fancyfoot[C]{} % Empty center footer
\fancyfoot[R]{\thepage} % Page numbering for right footer
\renewcommand{\headrulewidth}{0pt} % Remove header underlines
\renewcommand{\footrulewidth}{0pt} % Remove footer underlines
\setlength{\headheight}{13.6pt} % Customize the height of the header

\numberwithin{equation}{section} % Number equations within sections (i.e. 1.1, 1.2, 2.1, 2.2 instead of 1, 2, 3, 4)
\numberwithin{figure}{section} % Number figures within sections (i.e. 1.1, 1.2, 2.1, 2.2 instead of 1, 2, 3, 4)
\numberwithin{table}{section} % Number tables within sections (i.e. 1.1, 1.2, 2.1, 2.2 instead of 1, 2, 3, 4)

\setlength\parindent{0pt} % Removes all indentation from paragraphs - comment this line for an assignment with lots of text

%----------------------------------------------------------------------------------------
%	TITLE SECTION
%----------------------------------------------------------------------------------------

\newcommand{\horrule}[1]{\rule{\linewidth}{#1}} % Create horizontal rule command with 1 argument of height

\title{	
\normalfont \normalsize 
\textsc{Digital Signal Processing} \\ [25pt] % Your university, school and/or department name(s)
\horrule{0.5pt} \\[0.4cm] % Thin top horizontal rule
\huge Problem Set 1 \\ % The assignment title
\horrule{2pt} \\[0.5cm] % Thick bottom horizontal rule
}

\author{John McCormack} % Your name

\date{\normalsize\today} % Today's date or a custom date

\begin{document}

\maketitle % Print the title

%----------------------------------------------------------------------------------------
%	PROBLEM 1
%----------------------------------------------------------------------------------------

\section{Problem 1}

\subsection{Part a}
	If the highest frequency we need is 100 Hz, then we need a Nyquist Frequency
	greater than 200 Hz in order to properly sample it.

\subsection{Part b}
	If we are sampling a signal at 250 samples/s, then we can sample 
	up to 125 Hz and still have a unique sginal. 

\section{Problem 2}

Given the two equations:

\begin{align} 
\begin{split}
	x_{a}(t)	&= sin(480 \pi t) + 3sin(720 \pi t) \\
	x_{b}(t)	&= 3cos(600 \pi t) + 2cos(1800 \pi t)\\
\end{split}					
\end{align}

\subsection{Part a}
	The Nyquist Sampling frequency will be based off of the higher of the two frequencies. 
	Therefore, in part a the nyquist frequency is:
	\begin{align}
	\begin{split}
		x_{a}(t)	&= 3sin(720 \pi t)\\
		highest \: frequency  &= 360 \: Hz\\
		Nyquist \: frequency &= 720 \: samples/second\\
	\end{split}
	\end{align}

	And in equation b:
	\begin{align}
	\begin{split}
		x_{b}(t)	&= 2cos(1800 \pi t)\\
		highest \:  frequency &= 900 \: Hz\\
		Nyquist \:  frequency &= 1800 \: samples/second
	\end{split}
	\end{align}

\subsection{Part b}
	If the sampled signals are passes through an ideal D/A converter, the reconstructed
	signals would appear as below:

	\begin{align}
	\begin{split}
		x_{a}(t)	&= sin(480 \pi t) + 3sin(720 \pi t)\\
		t &= \frac{n}{Fs} \\
		Fs &= 600 \\
		x_{a}(n) &= sin(480 \pi \frac{n}{600}) + 3sin(720 \pi \frac{n}{600}) \\
		x_{a}(n) &= sin(\frac{4}{5} \pi n) + 3sin(\frac{6}{5} \pi n) \\
		x_{a}(n) &= sin(2 \pi n \frac{2}{5}) + 3sin(2 \pi n \frac{3}{5}) \\
		x_{a}(n) &= sin(2 \pi n \frac{2}{5}) + 3sin(2 \pi n (1 - \frac{2}{5})) \\
		x_{a}(n) &= sin(2 \pi n \frac{2}{5}) - 3 sin(2 \pi n \frac{2}{5}) \\
		Recover \: Signals \\
		y_{a}(t) &= -2sin(2 \pi t \frac{2}{5} * 600)\\
		y_{a}(t) &= -2sin(2 \pi 240 t)) \\
	\end{split}
	\end{align}

	The two sinewaves appear as one signal due to aliasing in the second part of the signal. 

	For the Other Signal:
	\begin{align}
	\begin{split}
		x_{b}(t)	&= 3cos(600 \pi t) + 2cos(1800 \pi t) \\
		t &= \frac{n}{Fs} \\
		Fs &= 10,000 \\
		x_{b}(n) &= 3 cos(600 \pi \frac{n}{10000}) + 2cos(1800 \pi \frac{n}{10000}) \\
		x_{b}(n) &= 3 cos(\pi n \frac{3}{50}) + 2 cos(\pi n \frac{9}{50}) \\
		Recover \: Signals \\
		y_{b}(t) &= 3 cos(\pi t 600) + 2 cos(\pi t 1800) \\
	\end{split}
	\end{align}

	The two parts of the signal are recovered successfully

\section{Part 3}

If we have a signal:
	\begin{align}
		\begin{split}
			x(t) = cos(4000 \pi t) \\
		\end{split}
	\end{align}

	That was sampled to produce 

	\begin{align}
		\begin{split}
			x(n) = cos(n\pi / 2) \\
		\end{split}
	\end{align}

	Then the sampling rate, measured in Hz is:

	\begin{align}
		\begin{split}
			t &= \frac{n}{Fs} \\
			cos(4000 \pi t) &= cos(n \pi /2) \\
			cos(4000 \pi \frac{n}{Fs}) &= cos(n \pi /2) \\
			4000 \pi \frac{n}{Fs} &= \frac{n \pi}{2} \\
			\frac{4000}{Fs} &= \frac{1}{2} \\
			Fs &= 8000 \: Hz \\
		\end{split}
	\end{align}

\end{document}
